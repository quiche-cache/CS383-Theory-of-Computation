\documentclass[11pt]{article}
\usepackage{fullpage}
\usepackage{algorithm}
\usepackage[noend]{algorithmic}
\usepackage{amsmath,amssymb,amsthm}
\usepackage{stackrel}
\usepackage{enumerate}

% These define new environments / formats for lemmas, definitions, running time, etc.
\newtheorem{lemma}{Lemma}
\newtheorem{definition}{Definition}
\newtheorem{notation}{Notation}
\newtheorem*{claim}{Claim}
\newtheorem{observation}{Observation}
\newtheorem{conjecture}[lemma]{Conjecture}
\newtheorem{theorem}[lemma]{Theorem}
\newtheorem{corollary}[lemma]{Corollary}
\newtheorem{proposition}[lemma]{Proposition}
\newtheorem*{rt}{Running Time}

\newcommand{\bc}[1]{{\quad \text{(#1)}}} 	% Justification in math env
\newcommand{\st}{{\text{ such that }}} 							   % Math env
\newcommand{\abs}[1]{{\left |#1\right |}} 							% Absolute value / Cardinality
\newcommand{\bld}[2]{\noindent\textbf{#1:}\hspace{0.1in}#2$  $\bigskip} % Headings
\newcommand{\notimplies}{%
	\mathrel{{\ooalign{\hidewidth$\not\phantom{=}$\hidewidth\cr$\implies$}}}}

% These define nice ways to format P and OPT (use \P or \opt)
\def\P{\ensuremath{$ \mathcal{P} $}}
\def\opt{\ensuremath{\textsc{opt}}}
\renewcommand{\labelenumi}{\bf \alph{enumi}.}
\newcommand{\Z}{\mathbb{Z}}
\newcommand{\linesep}{\noindent\bigskip\rule{17cm}{0.1mm}\bigskip}

\renewcommand\maketitle{
	\begin{center}
		\begin{tabular*}{6.44in}{l @{\extracolsep{\fill}}c r}
			\bfseries  &  & \bfseries CSCI 383 Spring 2019 \\
			\bfseries&  & \bfseries  Homework \#1 Solutions  \\
			\bfseries   &   &  \bfseries Kai Ting Keshia Yap\\ 
		\end{tabular*}
\end{center} }

\begin{document}
	\maketitle
	
	\noindent Honor Code: I affirm that I adhered to the Honor Code in this assignment. Keshia Yap\\
	
	\subsection*{Part  2: Sets}
	Let $ \Sigma $ be an alphabet of $ m > 0 $ symbols.
	\begin{enumerate}[1. ]
		\item Suppose $ \Sigma=\{a,b,c\} $. List the elements of $ \Sigma^3 $.
		\item Use mathematical induction to prove that for all $ n\geq 0, \abs{\Sigma^n}=m^n$.
		\item Find a closed formula for $ \abs{\Sigma^{\leq n}} $ where $ \Sigma^{\leq n} =\bigcup^n_{k=0} \Sigma^k$.
		\item Using the definitions of union and intersection on p10 of Kozen or in the notes, formally show that $ R\cup(S\cap T)= (R\cap S)\cup (R\cap T)$.
	\end{enumerate}
	\linesep
	\begin{enumerate}[1. ]
		\item aaa,aab,abb,aba,aac,acc,aca,abc,acb,\\
		bbb,bba,bab,baa,bcc,bac,bca,bac,bca,\\
		ccc,cca,cac,cab,ccb,cbc,cbbcba,cab\\
		
		
		\item Base case: when $ n=0 $, we know from class that $ \abs{\Sigma^n}=\abs{\{\varepsilon\}}=1=m^0$.\\
		Inductive case: Suppose that for some non-negative integer $ k $, $ \abs{\Sigma^k}=m^k $. \\
		Now, $ \abs{\Sigma^{k+1}}=\abs{\Sigma\Sigma^k}=\abs{\Sigma}\abs{\Sigma^k}=m\cdot m^k=m^{k+1} $. Therefore by mathematical induction,  $\abs{\Sigma^n}=m^n$ for all $ n\geq 0$.\\
		
		\item $ \abs{\Sigma^{\leq n}} =\abs{\Sigma^0\cup\Sigma^1\cup\dots\cup\Sigma^n}$. For non-negative integers $ k,j $, if $k\neq j$ then $ \Sigma^k \cap \Sigma^j=\empty$, so the union above is a disjoint union. (\textit{Proof}: If there exists a word in both $ \Sigma^j $ and $ \Sigma^k $ then the word has both length $ j $ and length $ k $, so $ j=k $ which contradicts our assumption.) 
		
		This implies that \[\abs{\Sigma^{\leq n}}=\abs{\Sigma^0}+\abs{\Sigma^1}+\dots+\abs{\Sigma^n}=m^0+m^1+m^2+\dots+m^n\] which we will call $ S $. Observe that $ mS= m^1+m^2+m^3\dots+m^{n+1}$, so $ S(1-m)=m^0-m^{n+1} $ and therefore $ \abs{\Sigma^{\leq n}}=S=\dfrac{1-m^{n+1}}{1-m} $. \\
		
		
		\item Since $ \lor $ is distributive over $ \land $, \begin{align*}
		x\in R\cup (S\cap T)&\iff (x\in R)\lor (x\in S\land x\in T) \\
		&\iff (x\in R \lor x\in S) \land (x\in R\lor x\in T)\\
		&\iff (x\in R\cup S)\land (x\in R\cup T)\\
		&\iff x\in (R\cup S)\cap (R\cup T)
		\end{align*}
	\end{enumerate}
\end{document}
