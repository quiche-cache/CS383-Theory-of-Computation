\documentclass[11pt]{article}
\usepackage{fullpage}
\usepackage{algorithm}
\usepackage[noend]{algorithmic}
\usepackage{enumerate}
\usepackage{amsmath,amssymb,amsthm}
\usepackage{stackrel}

% Helpful Shortcuts
\newcommand{\bc}[1]{{\quad \text{(#1)}}} 	% Justification in math env
\newcommand{\st}{{\text{ such that }}} 							   % Math env
\newcommand{\abs}[1]{{ |#1 |}} 							% Absolute value / Cardinality
\newcommand{\bld}[2]{\noindent\textbf{#1:}\hspace{0.1in}#2$  $\bigskip} % Headings
\newcommand{\notimplies}{%
	\mathrel{{\ooalign{\hidewidth$\not\phantom{=}$\hidewidth\cr$\implies$}}}}
\newcommand{\linesep}{\noindent\bigskip\rule{17cm}{0.1mm}\bigskip} % Horizontal Line

% These define new environments / formats for lemmas, definitions, running time, etc.
\newtheorem{lemma}{Lemma}
\newtheorem{definition}{Definition}
\newtheorem{notation}{Notation}
\newtheorem*{claim}{Claim}
\newtheorem{observation}{Observation}
\newtheorem{conjecture}[lemma]{Conjecture}
\newtheorem{theorem}[lemma]{Theorem}
\newtheorem{corollary}[lemma]{Corollary}
\newtheorem{proposition}[lemma]{Proposition}
\newtheorem*{rt}{Running Time}

% These define nice ways to format P and OPT (use \P or \opt)
\def\P{\ensuremath{$ \mathcal{P} $}}
\def\opt{\ensuremath{\textsc{opt}}}
\renewcommand{\labelenumi}{\bf \alph{enumi}.}

\renewcommand\maketitle{
	\begin{center}
		\begin{tabular*}{6.44in}{l @{\extracolsep{\fill}}c r}
			\bfseries  &  & \bfseries CSCI 383 Spring 2019 \\
			\bfseries&  & \bfseries  Homework \#6 Solutions  \\
			\bfseries   &   &  \bfseries Kai Ting Keshia Yap\\ 
		\end{tabular*}
\end{center} }

\begin{document}
	\maketitle
	
	\noindent Honor Code: I affirm that I adhered to the Honor Code in this assignment. Keshia Yap\\
	
	
	\subsection*{Part 3: Always Be Closing}
	Give CFGs for each of the following languages. You do not need to prove the correctness of your
	answer, but you should explain your answers enough that is easy for a reader to understand why
	your CFGs are correct.
	\begin{enumerate}
		\item $ L_4=\{x\in\{a,b\}^*\mid x $ is not a palindrome$ \} $
		\item $ L_5=\{x\in\{a,b\}^*\mid  \#a(x) \neq \#b(x) \} $
		\item $ L_6 = L_4 \cup L_5 $
		\item Inspired by your CFG for $ L_6 $, briefly argue that the set of CFLs is closed under union. No
		formal proof necessary.
		\item Briefly answer the following questions:
		– How does $ L_3 $ relate to $ L_5 $ and $ L_6 $?
		– What does this say about the closure of the set of CFLs under set complement?
	\end{enumerate}
	\linesep
	
	\noindent In this homework, we notate the complement of a set $ A $ to be $ A^c$.
	
	\begin{enumerate}
		\item We claim that $ G_4=(N, \Sigma, P_4, S_4) $ is the CFG for $ L_4 $, where
		\begin{itemize}
			\item $ N=\{S_4,D\} $
			\item $ \Sigma=\{a,b\} $
			\item $ S_4 \rightarrow aDb\mid bDa\mid aS_4b\mid bS_4a\mid aS_4a\mid bS_4b $
			\item $ D\rightarrow aDa\mid bDb\mid a\mid b\mid \varepsilon $
		\end{itemize}
		Explanation:  Observe that 
		\begin{enumerate}[1.]
			\item Sentential forms derivable from $ S_4 $ and $ D $ that are not single characters or epsilon add exactly one character to both sides of $ S_4 $ or of $ D $.
			\item Sentential forms derivable from $ D $ are palindromes, as shown in class.
			\item Sentential forms derivable from $ S_4 $ that only contain terminals must use the production $ S_4\rightarrow aDb\mid bDa $, since no strings comprising of only terminals are derivable from $ S_4 $ in just one step. 
		\end{enumerate}
		Therefore, the CFG produces strings of the form $x= u_1u_2\dots u_kvw_kw_{k-1}\dots w_1 $ where $ v\in\{a,b,\varepsilon\} $ and $ u_i,w_i\in\{a,b\}$  $ \forall i\in[1,k]$. From the third observation, there must exist some $ i\in [1,k] $ such that $ u_iw_i=ba $ or $ ab $, so $ x $ cannot be a palindrome.
		
		\item We claim that $ G_5=(N, \Sigma, P_5, S_5) $ is the CFG for $ L_5 $, where 
		\begin{itemize}
			\item $ N=\{S_5, A,B,E\} $
			\item $ \Sigma=\{a,b\} $
			\item $ S_5 \rightarrow A\mid B$
			\item $E\rightarrow aEb\mid bEa\mid EE\mid \varepsilon$
			\item $ A\rightarrow a\mid AE\mid EA\mid AA $
			\item $ B\rightarrow b\mid BE\mid EB\mid BB $
		\end{itemize}
		Explanation: First, notice that we can separate the strings in $ L_5 $ into two disjoint sets: those with more $ a $'s than $ b $'s, and those with more $ b $'s than $ a $'s. We call these two sets $ A' $ and $ B' $ respectively. Since the positive integers are closed under addition, the sets $ A' $ and $ B' $ are each closed under concatenation. \\
		
		Now let the set $ E' $ denote the set of strings with equal number of $ a $'s and $ b $'s. Clearly, $ E' =L_5^c$, and we saw in class that $ E $ produces strings in $ E' $. We observe that $ A'E'=A'=E'A' $ and $ B'E'=B'=E'B' $. Moreover, as base cases, $ a\in A' $ and $ b\in B' $. Therefore, the strings derivable from $ A $ (and $ B $) are strings in $ A' $ (and $ B' $ respectively).\\
		
		Since $ S_5 $ produces strings derivable from $ A $ or from $ B $, $ G_5 $ produces strings with unequal number of $ a $'s and $ b $'s.
		
		\item We claim that $ G=(N, \Sigma, P, S) $ is the CFG for $ L_6 $, where 
		 \begin{itemize}
			\item $ N=\{S, S_4, S_5, D, A, B, E\} $
			\item $ \Sigma=\{a,b\} $
			\item $ P=P_4\cup P_5 \cup \{S\rightarrow S_4 \mid S_5 \}$ (If a production is in $ P_4 $ or $ P_5 $, then it is in $ P $.)
		\end{itemize}
		Explanation: If $x$ is a string in $ L_4\cup L_5 $, then it has to be in at least one of $ L_4 $ or $ L_5 $. So it can be produced by at least one of $ G_4 $ or $ G_5 $. Therefore, $ G $ must contain the productions in both $ G_4 $ and $ G_5 $. Since $ G_4 $ and $ G_5 $ have different start states, for simplicity, we simply add a new start state $ S $ and add the production that $ S $ to the start states of $ G_4 $ and $ G_5 $. It is then trivial to see that $ N $ and $ \Sigma $ should respectively be the union of the non-terminals with $ \{S\} $, and the alphabets of $ G_4 $ and $ G_5 $.
		
		\item As seen in the above example, we can simply 'combine' the CFGs for two languages to get the CFG of the union of those languages, so the set of CFLs is closed under union. The reasoning is similar to that of part c.\\
		
		Formally, if $ G_1=(N_1, \Sigma_1, P_1, S_1) $ and $ G_2=(N_2, \Sigma_2, P_2, S_2) $ are CFGs for languages $ L_1 $ and $ L_2 $ respectively, then the CFG for $ L_1\cup L_2 $ is \[G=(N_1\cup N_2\cup\{S\}, \Sigma_1\cup \Sigma_2, P_1\cup P_2\cup \{S\rightarrow S_1\mid S_2\}, S)\]
		
		\item We observe that $ L_3=(L_5\cap L_6)^c $. This means that the set of CFLs is not closed under set complement, as $ L_3^c $ is closed but not $ L_3=(L_3^c)^c $.
	\end{enumerate}
	\end{document}