\documentclass[11pt]{article}
\usepackage{fullpage}
\usepackage{algorithm}
\usepackage[noend]{algorithmic}
\usepackage{amsmath,amssymb,amsthm}
\usepackage{stackrel}

% These define new environments / formats for lemmas, definitions, running time, etc.
\newtheorem{lemma}{Lemma}
\newtheorem{definition}{Definition}
\newtheorem{notation}{Notation}
\newtheorem*{claim}{Claim}
\newtheorem{observation}{Observation}
\newtheorem{conjecture}[lemma]{Conjecture}
\newtheorem{theorem}[lemma]{Theorem}
\newtheorem{corollary}[lemma]{Corollary}
\newtheorem{proposition}[lemma]{Proposition}
\newtheorem*{rt}{Running Time}

\newcommand{\bc}[1]{{\quad \text{(#1)}}} 	% Justification in math env
\newcommand{\st}{{\text{ such that }}} 							   % Math env
\newcommand{\abs}[1]{{ |#1 |}} 							% Absolute value / Cardinality

% These define nice ways to format P and OPT (use \P or \opt)
\def\P{\ensuremath{$ \mathcal{P} $}}
\def\opt{\ensuremath{\textsc{opt}}}
\renewcommand{\labelenumi}{\bf \alph{enumi}.}

% Horizontal line
\newcommand{\linesep}{\noindent\bigskip\rule{17cm}{0.1mm}}

\renewcommand\maketitle{
	\begin{center}
		\begin{tabular*}{6.44in}{l @{\extracolsep{\fill}}c r}
			\bfseries  &  & \bfseries CSCI 383 Spring 2019 \\
			\bfseries&  & \bfseries  Homework \#3 Solutions  \\
			\bfseries   &   &  \bfseries Kai Ting Keshia Yap\\ 
		\end{tabular*}
\end{center} }

\begin{document}
	\maketitle
	\noindent Honor Code: I affirm that I adhered to the Honor Code in this assignment. Keshia Yap\\
	
	\subsection*{Parp 3: Typo Time}
	For any language $ E $, we can define another language $ Nearly(E) $ which contains all strings that
	differ from some string in $ E $ by exactly one letter. For example, if $ \Sigma=\{a,b\} $ and $ E = \{aa, ab, aaa\} $ then $ Nearly(E) = \{ab, ba, aa, bb, aab, aba, baa\} $. Note that to generate new strings in $ Nearly(E) $, we change characters in strings from $ E $, but we do not remove or add letters. Prove that if E is regular, $ Nearly(E) $ is also regular.
	
	\linesep\\
	\noindent Since $ E $ is regular, there exists a DFA $ M=(Q,\Sigma, \delta, s, F) $ such that $ L(M)=E $. \\
	
	We want to show that there exists an NFA that accepts $ Nearly(E) $. The idea behind its construction having two copies of the DFA, the second one called $ M'=(Q',\Sigma, \delta', s', F')  $, where $ F'=\{q'\mid q\in F\}  $ and $ Q'=\{q'\mid q\in Q\}  $. Add in-between states that brings a state in $ Q $ to the corresponding `next' state in $ Q' $ for every transition in $ M $ , via a character not in the original transition.
	
	\noindent Let $ M_N=(Q_N, \Sigma_N, \Delta, S, F_N) $, where $ F_N=F'$, $ S=\{s\} $,
	
	\begin{itemize}
		\item $ Q_N=Q\cup \{q_{a}, q_{a}'\mid q\in Q, a\in \Sigma \} \cup Q'$
		\item $ \Delta(q,c)=\{\delta(q,c)\} $, as in $M$
		\item $ \Delta(q',c)=\{\delta(q,c)'\} $, as in $M'$
		\item $ \Delta(q,\varepsilon)=\{q_a \mid a\in \Sigma\} $
		\item $ \Delta(q_a, c)=\{q_a'\mid c\in\Sigma,  c\neq a \}$
		\item $ \Delta(q_a', \varepsilon)=\{\delta(q,a)'\}$
	\end{itemize}
	
	\noindent In the proofs of the following lemmas, recall that by Problem 1 of Homework 3, we can write \[\hat{\Delta}(A,x')= \hat{\Delta}(A,ubw)= \hat{\Delta}(\hat{\Delta}(A,ub),w)= \hat{\Delta}(\hat{\Delta}(\hat{\Delta}(A,u),b),w)\] for $ A\subseteq Q_N $.\bigskip
	
	\begin{lemma}
		Let $ u\in\Sigma^* $. Then $ \hat{\delta}(s,u)\in\hat{\Delta}(S,u) $.
	\end{lemma}
	\begin{proof}
		Base case: $ u=\varepsilon $ \\
		From the definitions above, $ \hat{\Delta}(S,u)=\hat{\Delta}(S,\varepsilon)=S =\{s\}\ni s  \hat{\delta}(s,\varepsilon)=\hat{\delta}(s,u) $.\\
		
		\noindent Inductive case: $ u=vc $\\
		$ \hat{\Delta}(S,u)=\hat{\Delta}(S,vc)=\Delta(\hat{\Delta}(S,v),c)\ni \delta(\hat{\delta}(s,v),c)=\hat{\delta}(s,vc)= \hat{\delta}(s,u)$.\\
	\end{proof}
	\begin{lemma}
	Let $ u\in\Sigma^* , a\neq b\in\Sigma$. Then $ \hat{\delta}(s,ua)'\in  \hat{\Delta}(S,ub)$
	\end{lemma}
	\begin{proof}
		\begin{align*}
			\hat{\Delta}(S,ub)&=\hat{\Delta}(s,ub)&&\bc{$S=\{s\}$}\\
			&=\Delta(\hat{\Delta}(S,u),b)&&\bc{def of $\hat{\Delta}$}\\
			&=\Delta(\Delta(\hat{\Delta}(S,u),\varepsilon),b)&&\bc{def of $\hat{\Delta}$}\\
			&\supseteq \Delta(\Delta(\hat{\delta}(s,u),\varepsilon),b)&&\bc{Lemma 1}\\
			&\supseteq\Delta(\hat{\delta}(s,u)_a',b)&&\bc{def of $\Delta(q,\varepsilon)$}\\
			%&\supseteq\hat{\delta}(s,u)_a'&&\bc{def of $\Delta(q_a',b),a\neq b$}\\
			&\ni \Delta(\Delta(\hat{\delta}(s,u)_a',\varepsilon),b)&&\bc{def of $\Delta$}\\
			&\ni \Delta(\delta(\hat{\delta}(s,u),a)')&&\bc{def of $\Delta(q_c', \varepsilon)$}\\
			&=\hat{\delta}(s,ua)&&\bc{def of $\hat{\delta}$}
		\end{align*}		
	\end{proof}
	\begin{lemma}
		Let $ u,w\in\Sigma^* , a\neq b\in\Sigma$. Then $ \hat{\delta}(s,uaw)'\in \hat{\Delta}(S,ubw)$
	\end{lemma}
	\begin{proof}
	    \noindent Base case: $ w=\varepsilon $\\
	    \begin{align*}
		    \hat{\delta}(s,uaw)'&=\hat{\delta}(s,ua)'&&\bc{$w=\varepsilon$}\\
		    &\in\hat{\Delta}(S,ub)&&\bc{Lemma 2}\\
		    &=\hat{\Delta}(S,ubw)&&\bc{$w=\varepsilon$}
	    \end{align*}
		
		\noindent Inductive case: $ w=vc $\\
		\begin{align*}
		    \hat{\Delta}(S,ubw)&=\hat{\Delta}(S,ubvc) &&\bc{$w=vc$}\\
		    &=\Delta(\hat{\Delta}(S,ubv),c)&&\bc{def of $\hat{\Delta}$}\\
		    &=\Delta(\hat{\Delta}(s,ubv),c)&&\bc{$S=\{s\}$}\\
		    &\supseteq \Delta(\hat{\delta}(s,uav)',c)&&\bc{inductive hypothesis}\\
		    &=\{\delta(\hat{\delta}(s,uav),c)'\}&&\bc{def $\Delta(q',c)$}\\
		    &=\{\hat{\delta}(s,uavc)'\}&&\bc{def of $\hat{\delta}$}\\
		    &\ni \hat{\delta}(s,uaw)'&&\bc{$w=vc$}
		\end{align*}
	\end{proof}		

	
	\begin{theorem}
		$ L(M_N) = Nearly(E)$.
	\end{theorem}
	\begin{proof}
		Let $ x\in L(M) $. We can write $ x=uaw $ where $ u,w\in \Sigma^*, a\in \Sigma $. We want to show that any string that differs from $ x $ by exactly one character, denoted $ x':=ubw$ where $ a\neq b\in\Sigma $. That is,
		
		\[\hat{\delta}(s,x)\in F\implies \hat{\Delta}(S, x')\cap F_N\neq \varnothing. \]
		

		\noindent By Lemma 3, $\hat{\Delta}(S,ubw)\ni \hat{\delta}(s,uaw)' = \hat{\delta}(s,x)'\in F'=F_N $. Therefore $ \hat{\Delta}(S, x')\cap F_N\neq \varnothing $.
	\end{proof}
		
	We showed in class that we recursively remove $\varepsilon$-transitions convert any NFA to a DFA in finitely many steps. Therefore, we can convert $ M_N $ into a DFA that accepts the language $ Nearly(E) $. Therefore, if $ E $ is regular, $ Nearly(E) $ is also regular. 
	
	
		
\end{document}
