\documentclass[11pt]{article}
\usepackage{fullpage}
\usepackage{algorithm}
\usepackage[noend]{algorithmic}
\usepackage{enumerate}
\usepackage{amsmath,amssymb,amsthm}
\usepackage{stackrel}

% Helpful Shortcuts
\newcommand{\bc}[1]{{\quad \text{(#1)}}} 	% Justification in math env
\newcommand{\st}{{\text{ such that }}} 							   % Math env
\newcommand{\abs}[1]{{ |#1 |}} 							% Absolute value / Cardinality
\newcommand{\bld}[2]{\noindent\textbf{#1:}\hspace{0.1in}#2$  $\bigskip} % Headings
\newcommand{\notimplies}{%
	\mathrel{{\ooalign{\hidewidth$\not\phantom{=}$\hidewidth\cr$\implies$}}}}
\newcommand{\linesep}{\noindent\bigskip\rule{17cm}{0.1mm}\bigskip} % Horizontal Line
\newcommand{\Z}{{\mathbb{Z}}}

% These define new environments / formats for lemmas, definitions, running time, etc.
\newtheorem{lemma}{Lemma}
\newtheorem{definition}{Definition}
\newtheorem{notation}{Notation}
\newtheorem*{claim}{Claim}
\newtheorem{observation}{Observation}
\newtheorem{conjecture}[lemma]{Conjecture}
\newtheorem{theorem}[lemma]{Theorem}
\newtheorem{corollary}[lemma]{Corollary}
\newtheorem{proposition}[lemma]{Proposition}
\newtheorem*{rt}{Running Time}

% These define nice ways to format P and OPT (use \P or \opt)
\def\P{\ensuremath{$ \mathcal{P} $}}
\def\opt{\ensuremath{\textsc{opt}}}
\renewcommand{\labelenumi}{\bf \alph{enumi}.}

\renewcommand\maketitle{
	\begin{center}
		\begin{tabular*}{6.44in}{l @{\extracolsep{\fill}}c r}
			\bfseries  &  & \bfseries CSCI 383 Spring 2019 \\
			\bfseries&  & \bfseries  Homework \#7 Solutions  \\
			\bfseries   &   &  \bfseries Kai Ting Keshia Yap\\ 
		\end{tabular*}
\end{center} }

\begin{document}
	\maketitle
	
	\noindent Honor Code: I affirm that I adhered to the Honor Code in this assignment. Keshia Yap\\
	
	
	\subsection*{Part 2: osure}
	Show using PDAs that if $ A $ is a context-free language, then
	$ Suffix(A) = \{v \mid uv \in A$ for some string u  $\in\Sigma^*\} $
	is also context-free. In other words, you should:
	\begin{itemize}
	\item Informally describe a procedure for turning an NPDA for $ A $ into an NPDA for $ Suffix(A) $.
	\item Formally specify your procedure.
	\item Prove that your construction is correct.
	\end{itemize}
\linesep

	If $ A $ is a context-free language, then there exists some NPDA $ M=(\{p\},\Sigma, \Gamma, \delta, p, \bot) $ with a single state $ p $ and accepts strings by empty stack such that $ L(M)=A $. Let $ M'=(Q,\Sigma, \Gamma, \delta', s, \bot)  $ be an NPDA where: 
	\begin{itemize}
		\item $ Q=\{s,p\}$
		\item $ \delta\subseteq \delta' $
		\item For all $ \alpha\in \Gamma, \beta\in \Gamma'$ and $ c\in\Sigma\cup\{\varepsilon\} $, if $ ((p,c,\alpha),(p,\beta) \in \delta$, then $ ((s,\epsilon,\alpha),(s,\beta)) \in \delta'$.
		\item For all $ \alpha\in\Gamma $, $ ((s,\epsilon, \alpha),(p,\alpha)\in\delta' $.\\
	\end{itemize}

	This machine $ M '$ contains the original machine $ M $ but has some add-ons. Namely, it has a new state $ s $ that has transitions similar to those in $ M $, but instead of reading a character during each transition, we allow it to make the same transitions without reading any characters (or as if it's reading `ghost characters'). \\
	
	Intuitively, this machine would be able to make the same transitions as if it read the string $ u $, but without actually reading it. Therefore, given the suffix $ v $, the machine can use the same transitions as if it were reading the string $ u $ first to have the same stack that $ M $ would have after reading the string $ u $. \\
	
	The machine can transition to the state $ p $ at any time, after which it must read characters as it did in $ M $. After making this transition, it can read the string $ v $ and have the stack emptied. (It cannot stay in state $ s $ as $ s $ only contains $ \varepsilon $ transitions that would not read the string.)\\
	
	\newpage
	\begin{lemma}
		For all $ \gamma\in\Gamma^* $, \[\exists x\in\Sigma^*, (p,x,\bot)\xrightarrow[M]{*} (p,\varepsilon, \gamma) \iff (s,\varepsilon,\bot)\xrightarrow[M']{*} (s,\varepsilon, \gamma). \]

%		For all $ x\in\Sigma^*, \gamma\in\Gamma^*$, if $ (p,x,\bot)\xrightarrow[M]{*} (p,\varepsilon, \gamma) $ then $ (s,\varepsilon,\bot)\xrightarrow[M']{*} (s,\varepsilon, \gamma) $. Conversely, if $ (s,\varepsilon,\bot)\xrightarrow[M']{*} (s,\varepsilon, \gamma) $ then $ \exists x\in\Sigma^*\st (p,x,\bot)\xrightarrow[M]{*} (p,\varepsilon, \gamma) $.
		
%		If $ \exists x\in\Sigma^*, \gamma\in\Gamma^*, n\in\Z^+\cup\{0\} $ such that $ (p,x,\bot)\xrightarrow[M]{n} (p,\varepsilon, \gamma) $ then $ (s,\varepsilon,\bot)\xrightarrow[M']{n} (s,\varepsilon, \gamma) $. The converse is also true.
	\end{lemma}
	\begin{proof}
		We proceed by induction. For the base case, suppose $ \gamma=\bot $. Then $ (s,\varepsilon,\bot)\xrightarrow[M']{0} (s,\varepsilon, \bot) $ is trivially true at all times. For the backward direction, take $ x=\varepsilon $.\\
		
		\noindent For the inductive hypothesis, suppose that \[\exists x\in\Sigma^*, (p,x,\bot)\xrightarrow[M]{*} (p,\varepsilon, \gamma) \iff (s,\varepsilon,\bot)\xrightarrow[M']{*} (s,\varepsilon, \gamma).\] 
		
		\noindent Then by construction, for all $ \alpha\in\Gamma, \beta\in\Gamma^*$, \[\exists c\in\Sigma, ((p,c,\alpha),(p,\beta))\in\delta \iff ((s,\varepsilon,\alpha),(s,\beta))\in\delta'.\] 
		
		\noindent So for all $ \alpha\in\Gamma, \beta\in\Gamma^*, $ if  $\exists\sigma\in\Gamma^*\st \gamma=\alpha\sigma$ and $ \gamma':=\beta\sigma $ then \[(p,xc,\bot)\xrightarrow[M]{*} (p,c, \gamma)\xrightarrow[M]{1} (p,c, \gamma')  \iff (s,\varepsilon,\bot)\xrightarrow[M']{*} (s,\varepsilon, \gamma) \xrightarrow[M']{1} (s,\varepsilon, \gamma').\]
		
%		For the base case, suppose that $ n=0 $. Then $ (p,x,\bot)\xrightarrow[M]{0} (p,\varepsilon, \gamma)  $, which implies that $ \bot=\gamma $ and $ x=\varepsilon $. Then trivially we have $ (s,x,\bot)\xrightarrow[M]{0} (p,\varepsilon, \gamma)  $. The converse is true by the same argument run backwards. \\
		
%		As the base case, suppose $ x\in\Sigma $. Then $ (p,x,\bot)\xrightarrow[M]{n} (p,\varepsilon, \gamma)  $ implies that there exists some transition $ ((p,x,\bot),(p,\gamma))\in\delta $, where $ n=1 $. By our construction, there exists the transition $ ((s,\varepsilon,\bot),(s,\gamma))\in\delta' $, so $ (s,\varepsilon,\bot)\xrightarrow[M']{n} (p,\varepsilon, \gamma) $.\\
		
%		Inductive case: Suppose that $ (p,x,\bot)\xrightarrow[M]{n} (p,\varepsilon, \gamma) \implies (s,\varepsilon,\bot)\xrightarrow[M]{n} (s,\varepsilon, \gamma). $ Then given $ \exists c\in\Sigma $, such that $ (p,xc,\bot)\xrightarrow[M]{m} (p,\varepsilon, \gamma') $, there must exist the transition $ ((p,c,\sigma),(p,\varepsilon,\theta))\in\delta $ where $ \sigma \in\Gamma$ and $ \gamma=\sigma\mu $ for some $ \mu\in\Gamma^* $, and $ \gamma'=\theta\mu $. This implies that there exists the transition $ ((s,\varepsilon,\sigma),(p,\varepsilon,\theta))\in\delta' $. So $  (s,\varepsilon,\bot)\xrightarrow[M']{n} (s,\varepsilon, \gamma)\xrightarrow[M']{1}(s,\varepsilon,\gamma') $.\\
	\end{proof}
	\begin{claim}
		$ L(M')=Suffix(A) $
	\end{claim}
	\begin{proof}
		Given $ x=uv\in A $, there exists transitions in $ M $ such that 
		\begin{align*}
			(p,x,\bot)\xrightarrow[M]{n} (p,\varepsilon,\varepsilon) &\iff \exists \gamma\in\Gamma^* \st \\
			&\phantom{\iff} (p,uv,\bot)\xrightarrow[M]{*}(p,v,\gamma)\xrightarrow[M]{*}(p,\varepsilon,\varepsilon)&&\bc{by substitution}\\
			&\iff (p,u,\bot)\xrightarrow[M]{*}(p,\varepsilon,\gamma)\\
			&\iff (s,\varepsilon,\bot)\xrightarrow[M']{*}(s,\varepsilon,\gamma) &&\bc{by Lemma 1}\\
			&\iff (s,v,\bot)\xrightarrow[M']{*}(s,v,\gamma)\\
			&\phantom{\iff} \xrightarrow[M']{1}(p,v,\gamma)\xrightarrow[M']{*}(p,\varepsilon,\varepsilon) &&\bc{$ \forall \alpha\in\Gamma, ((s,\varepsilon,\alpha),(p,\alpha))\in\delta' $}\\
			&\iff v\in L(M')
		\end{align*}
	\end{proof}
\end{document}