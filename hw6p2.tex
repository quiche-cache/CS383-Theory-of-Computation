\documentclass[11pt]{article}
\usepackage{mathtools}
\usepackage{fullpage}
\usepackage{algorithm}
\usepackage[noend]{algorithmic}
\usepackage{enumerate}
\usepackage{amsmath,amssymb,amsthm}
\usepackage{stackrel}

% Helpful Shortcuts
\newcommand{\bc}[1]{{\quad \text{(#1)}}} 	% Justification in math env
\newcommand{\st}{{\text{ such that }}} 							   % Math env
\newcommand{\abs}[1]{{ |#1 |}} 							% Absolute value / Cardinality
\newcommand{\bld}[2]{\noindent\textbf{#1:}\hspace{0.1in}#2$  $\bigskip} % Headings
\newcommand{\notimplies}{%
	\mathrel{{\ooalign{\hidewidth$\not\phantom{=}$\hidewidth\cr$\implies$}}}}
\newcommand{\linesep}{\noindent\bigskip\rule{17cm}{0.1mm}\bigskip} % Horizontal Line

% These define new environments / formats for lemmas, definitions, running time, etc.
\newtheorem{lemma}{Lemma}
\newtheorem{definition}{Definition}
\newtheorem{notation}{Notation}
\newtheorem*{claim}{Claim}
\newtheorem{observation}{Observation}
\newtheorem{conjecture}[lemma]{Conjecture}
\newtheorem{theorem}[lemma]{Theorem}
\newtheorem{corollary}[lemma]{Corollary}
\newtheorem{proposition}[lemma]{Proposition}
\newtheorem*{rt}{Running Time}

% These define nice ways to format P and OPT (use \P or \opt)
\def\P{\ensuremath{$ \mathcal{P} $}}
\def\opt{\ensuremath{\textsc{opt}}}
\renewcommand{\labelenumi}{\bf \alph{enumi}.}

\renewcommand\maketitle{
	\begin{center}
		\begin{tabular*}{6.44in}{l @{\extracolsep{\fill}}c r}
			\bfseries  &  & \bfseries CSCI 383 Spring 2019 \\
			\bfseries&  & \bfseries  Homework \#6 Solutions  \\
			\bfseries   &   &  \bfseries Kai Ting Keshia Yap\\ 
		\end{tabular*}
\end{center} }

\begin{document}
	\maketitle
	
	\noindent Honor Code: I affirm that I adhered to the Honor Code in this assignment. Keshia Yap\\
	
	\subsection*{Part 2: Zeroing In}
	Let $ A = \{w \in \{0, 1\}^*: $ the length of $ w $ is odd and its middle symbol is a $ 0\}$ and let $ G $ be the following grammar, denoted $ G: S \rightarrow 0 \mid 0S0 \mid 0S1 \mid 1S0 \mid 1S1 $
	\begin{enumerate}
	\item Convert $ G $ to Chomsky normal form.
	\item Prove by induction that $ L(G) = A $, that is, show that $ A \supseteq L(G) $ and $ L(G) \subseteq A $.
	\end{enumerate}
	\linesep
	
	\begin{enumerate}
		\item 
		\begin{itemize}
		\item $ \Sigma=\{0,1\} $
		\item $S\rightarrow 0\mid AB$  
		\item  $A\rightarrow 0\mid 1$  
		\item $B\rightarrow SC$  
		\item $C\rightarrow 0\mid 1$
		\end{itemize}
		\item \begin{proof}
		First suppose that $ w\in A$. By the definition, there exists some nonnegative integer $ m $  such that $ w=u0v$ where $u,v\in\{0,1\}^m$. We want to show that $S\xrightarrow[G]{*}w$.
		
		We claim that $S\xrightarrow[G]{2k}\alpha_k S \beta_k$ where $ \alpha_k=\underbrace{AA\dots AA}_{k\text{ times}} $ and $ \beta_k=\underbrace{BB\dots BB}_{k\text{ times}} $.\\
		
		\textit{Proof of claim: }We proceed by induction.
		\begin{itemize}
			\item Base case: $ k=0 $. It is trivial to see that $ S\xrightarrow[G]{0} S =\alpha_0S\beta_0$.
			\item Inductive hypothesis: There exists some nonnegative integer $ n $ such that $S\xrightarrow[G]{2n}\alpha_n S \beta_n$.
			\item Inductive case: \[S\xrightarrow[G]{2n}\alpha_n S \beta_n \xrightarrow[G]{1} \alpha_n AB\beta_n  \xrightarrow[G]{1} \alpha_n ASC \beta_n\]\\
		which is $ \alpha_{n+1} S\beta_{n+1} $. So $S\xrightarrow[G]{2(n+1)}\alpha_{n+1} S \beta_{n+1}$.\\
		\end{itemize}
		Suppose $ u=u_1u_2\dots u_m$ and $ v=v_1v_2\dots v_m  $, where $ u_i,v_i\in\{0,1\}\forall i\in[1,m] $. Then we can convert the first $ A $ to $ u_1 $, the second $ A $ to $ u_2 $ and so on, and then convert the first $ C $ to $ v_1 $, the second $ C $ to $ v_2 $ and so on. Since there are $ m $ $ A $'s and $ m $ $ C $'s, $\alpha_{m} S \beta_{m}\xrightarrow[G]{2m}uSv$. Finally, $ uSv\xrightarrow[G]{1}u0v=w $. So $ S\xrightarrow[G]{4m+1}w $. This implies that $ w\in L(G) $, so $ A \subseteq L(G) $.\bigskip
		
		Now, let $ w\in L(G) $. This means that $ w\in\{0,1\}^* $ such that $ S\xrightarrow[G]{*}w$. We want to show that there exists strings $ u,v $ such that $ w=u0v $ and $ \abs{u}=\abs{v} $.\\
		
		Claim: for any sentential form $ \gamma \st S\xrightarrow[G]{*}\gamma $, either $ \gamma= \alpha_mS\beta_m $ for some nonnegative integer $ m $, or $ \gamma =  \alpha_{m}AB\beta_m$, where $ \alpha_m\in \{0,1,A\}^m,\beta_m \in \{0,1,C\}^m$. \\
		
		\textit{Proof of claim: } We proceed by induction.\\
		\begin{itemize}
		\item Base case: Suppose that the first production we take is $ S\rightarrow 0 $. Since there are no more non-terminals on the right-hand side, we are done and $ \gamma=0 $, which is in the form $ \alpha_mS\beta_m $ where $ \alpha_{m}=\beta_{m}=\varepsilon $ (so $ m=0 $). Otherwise, we take $ S\rightarrow AB $ which is in the form $ \alpha_mAB\beta_m $ where $ m=0 $.\\
		
		\item Inductive hypothesis: Suppose that our claim is true for some integer $ k $. That is, for any sentential form $ \gamma \st S\xrightarrow[G]{*}\gamma $, either $ \gamma= \alpha_kS\beta_k $ for some nonnegative integer $ k $, or $ \gamma =  \alpha_{k}AB\beta_k$.\\
		
		\item Inductive case: Case 1: If $ \gamma= \alpha_kS\beta_k $, then $ \gamma \xrightarrow[G]{1} \alpha_kAB\beta_k $.\\
		
		Case 2: If $ \gamma= \alpha_{k}AB\beta_k $, then $ \gamma \xrightarrow[G]{1} \alpha_{k}ASC\beta_k = \alpha_{k+1}S\beta_{k+1}$.\\
		\end{itemize}
		
		The proof for Case 2 also implies that any sentential form derivable from $ S $ can be converted into the form $\alpha_{m}S\beta_m$ for some integer $ m $. Given that $ A,C $ only produce 0 or 1, and $ S $ only produces the terminal 0, converting $ \gamma $ to only non-terminals results in the expression $ w $ where $ w $ must be in $ \{0,1\}^m0\{0,1\}^m $ for some integer $ m $. So there must exist strings $ u,v \in \{0,1\}^m$ such that $ w=u0v $. By definition, $ \abs{u}=\abs{v}=m $.\\
		
		Therefore,  $ w \in A $, so $ L(G) \subseteq A $. Hence $ L(G) = A $.
		\end{proof}
	\end{enumerate}
\end{document}