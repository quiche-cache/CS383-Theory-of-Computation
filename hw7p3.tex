\documentclass[11pt]{article}
\usepackage{fullpage}
\usepackage{algorithm}
\usepackage[noend]{algorithmic}
\usepackage{enumerate}
\usepackage{amsmath,amssymb,amsthm}
\usepackage{stackrel}

% Helpful Shortcuts
\newcommand{\bc}[1]{{\quad \text{(#1)}}} 	% Justification in math env
\newcommand{\st}{{\text{ such that }}} 							   % Math env
\newcommand{\abs}[1]{{ |#1 |}} 							% Absolute value / Cardinality
\newcommand{\bld}[2]{\noindent\textbf{#1:}\hspace{0.1in}#2$  $\bigskip} % Headings
\newcommand{\notimplies}{%
	\mathrel{{\ooalign{\hidewidth$\not\phantom{=}$\hidewidth\cr$\implies$}}}}
\newcommand{\linesep}{\noindent\bigskip\rule{17cm}{0.1mm}\bigskip} % Horizontal Line

% These define new environments / formats for lemmas, definitions, running time, etc.
\newtheorem{lemma}{Lemma}
\newtheorem{definition}{Definition}
\newtheorem{notation}{Notation}
\newtheorem*{claim}{Claim}
\newtheorem{observation}{Observation}
\newtheorem{conjecture}[lemma]{Conjecture}
\newtheorem{theorem}[lemma]{Theorem}
\newtheorem{corollary}[lemma]{Corollary}
\newtheorem{proposition}[lemma]{Proposition}
\newtheorem*{rt}{Running Time}

% These define nice ways to format P and OPT (use \P or \opt)
\def\P{\ensuremath{$ \mathcal{P} $}}
\def\opt{\ensuremath{\textsc{opt}}}
\renewcommand{\labelenumi}{\bf \alph{enumi}.}

\renewcommand\maketitle{
	\begin{center}
		\begin{tabular*}{6.44in}{l @{\extracolsep{\fill}}c r}
			\bfseries  &  & \bfseries CSCI 383 Spring 2019 \\
			\bfseries&  & \bfseries  Homework \#7 Solutions  \\
			\bfseries   &   &  \bfseries Kai Ting Keshia Yap\\ 
		\end{tabular*}
\end{center} }

\begin{document}
	\maketitle
	
	\noindent Honor Code: I affirm that I adhered to the Honor Code in this assignment. Keshia Yap\\
	
	
	\subsection*{Part 3: Double Trouble}
	Let us define a Twin Pushdown Automaton (TPDA) to be like a standard Pushdown Automaton with two stacks, rather than one. In particular, the transition function would include transitions
	of the form
	$ (p, a, A, B) \rightarrow (q, \alpha, \beta) $ for $ p, q \in Q, a \in\Sigma\cup\{\varepsilon\}, A, B \in \Gamma $ and $ \alpha, \beta \in \Gamma^∗ $.\\
	
%	Such a rule would mean that while in state $ p $, if the character $ a $ is read from the string, $ A $ is top of the first stack, and $ B $ is on top of the second stack, then we can move to state $ q $, pop $ A $ and $ B $ from their respective stacks, and push $ \alpha $ and $ \beta $ onto the first and second stack respectively. The remainder of the definition of a TPDA would be analogous to that of a standard PDA. Acceptance is by final state. \\
	Prove that TPDAs and NPDAs are not equivalent in expressive power. 
%	In other words, find a language $ L $ that is not context-free, but for which there exists a TPDA. You do not need to prove that your TPDA accepts $ L $ (though you should provide a brief explanation and a formal description of your TPDA). You do need to prove that $ L $ is not context-free.
	
	\linesep
	
	We have shown in class that the language $ \{a^nb^na^n\mid n\geq 0\} $ is not context-free. However, we claim that there exists a TPDA $ M=\{Q,\Sigma, \Gamma, \delta, p, \bot, \bot, \{f\}\} $ where $ \bot $ is the bottom of stack symbol for both the first and the second stacks and $ L(M)= \{a^nb^na^n\mid n\geq 0\}$. 
	
	\begin{proof}
		We begin by defining the TPDA $ M $ that accepts by final state $ f $:
		\begin{itemize}
			\item $ Q=\{p,f\} $
			\item $ \Sigma = \{a,b\} $
			\item $ \Gamma=\{\bot, A\} $
			\item $ \delta $-transitions are given by:
			\begin{enumerate}[1.]
				\item $ ((p,a,\bot, \bot),(p,A\bot, A\bot)) $
				\item $ ((p,a,A,A),(p,AA,AA)) $
				\item $ ((p,b,A,A), (p,\varepsilon,A)) $
				\item $ ((p,a,\bot,A),(p,\bot,\varepsilon)) $
				\item $ ((p,\varepsilon, \bot,\bot),(f,\varepsilon,\varepsilon)) $
			\end{enumerate}
		\end{itemize}
	
	Observe that the first stack of the TPDA is equivalent to an NPDA that accepts the language $ \{a^nb^n\mid n\geq 0\} $. The second stack helps to keep track of the number of $ a $'s that were read initially and ensures that the same number of $ a $'s are read after the consecutive $ b $'s.\\
	
	We can analyze this in further detail. The TPDA adds an $ A $ to both stacks for every consecutive $ a $ read initially. After that, as each $ b $ is read, an $ A $ is removed from the first stack. If the number of consecutive $ b $'s read are the same as the number of initial consecutive $ a $'s, then only $ \bot $ should remain on the first stack. Subsequently, as each consecutive $ a $ is read, an $ A $ is removed from the second stack. If the number of consecutive $ a $'s read are the same as the number of initial consecutive $ a $'s, then the second stack should only contain $ \bot $. The machine is allowed to transition to the final state $ f $ if the top of the first and second stacks are both $ \bot $. So $ L(M)= \{a^nb^na^n\mid n\geq 0\}$. This implies that TPDAs and NPDAs are not equivalent in expressive power.
	\end{proof}
\end{document}