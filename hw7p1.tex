\documentclass[11pt]{article}
\usepackage{fullpage}
\usepackage{algorithm}
\usepackage[noend]{algorithmic}
\usepackage{enumerate}
\usepackage{amsmath,amssymb,amsthm}
\usepackage{stackrel}

% Helpful Shortcuts
\newcommand{\bc}[1]{{\quad \text{(#1)}}} 	% Justification in math env
\newcommand{\st}{{\text{ such that }}} 							   % Math env
\newcommand{\abs}[1]{{ |#1 |}} 							% Absolute value / Cardinality
\newcommand{\bld}[2]{\noindent\textbf{#1:}\hspace{0.1in}#2$  $\bigskip} % Headings
\newcommand{\notimplies}{%
	\mathrel{{\ooalign{\hidewidth$\not\phantom{=}$\hidewidth\cr$\implies$}}}}
\newcommand{\linesep}{\noindent\bigskip\rule{17cm}{0.1mm}\bigskip} % Horizontal Line

% These define new environments / formats for lemmas, definitions, running time, etc.
\newtheorem{lemma}{Lemma}
\newtheorem{definition}{Definition}
\newtheorem{notation}{Notation}
\newtheorem*{claim}{Claim}
\newtheorem{observation}{Observation}
\newtheorem{conjecture}[lemma]{Conjecture}
\newtheorem{theorem}[lemma]{Theorem}
\newtheorem{corollary}[lemma]{Corollary}
\newtheorem{proposition}[lemma]{Proposition}
\newtheorem*{rt}{Running Time}

% These define nice ways to format P and OPT (use \P or \opt)
\def\P{\ensuremath{$ \mathcal{P} $}}
\def\opt{\ensuremath{\textsc{opt}}}
\renewcommand{\labelenumi}{\bf \alph{enumi}.}

\renewcommand\maketitle{
	\begin{center}
		\begin{tabular*}{6.44in}{l @{\extracolsep{\fill}}c r}
			\bfseries  &  & \bfseries CSCI 383 Spring 2019 \\
			\bfseries&  & \bfseries  Homework \#7 Solutions  \\
			\bfseries   &   &  \bfseries Kai Ting Keshia Yap\\ 
		\end{tabular*}
\end{center} }

\begin{document}
	\maketitle
	
	\noindent Honor Code: I affirm that I adhered to the Honor Code in this assignment. Keshia Yap\\
	
	
	\subsection*{Part 1: Public Display of Aggression}
	\begin{enumerate}
	\item Give an NPDA for $ L = \{a^n b^{n+m}c^m\} $. You may choose to accept by final state or empty stack. No formal proof required, but give a clear, concise explanation.
	\item In your NPDA give the sequence of transitions you would apply to accept the string $ abbbcc $.
	\item Briefly argue that there is no sequence of transitions that would lead your machine to accept $ acbbbcc $.
	\end{enumerate}

	\linesep

	\begin{enumerate}
		\item Let $ M=(Q,\Sigma, \Gamma, \delta, s, \bot) $ be an NPDA that accepts by empty stack, where
		\begin{itemize}
			\item $ Q=\{s\} $
			\item $ \Sigma=\{a,b,c\} $
			\item $ \Gamma=\{\bot, A, B\} $
			\item $ \delta $-transitions are given by: 
			\begin{enumerate}[1.]
				\item $ ((s,a,\bot),(s,A\bot)) $
				\item $ ((s,a,A),(s,AA)) $
				\item $ ((s,b,A),(s,\varepsilon)) $
				\item $ ((s,b,\bot),(s,B\bot)) $
				\item $ ((s,b,B),(s,BB)) $
				\item $ ((s,c,B),(s,\varepsilon)) $
				\item $ ((s,\varepsilon,\bot),(s,\varepsilon)) $
			\end{enumerate}
		\end{itemize}
	\item The sequence of transitions are: 1,3,4,5,6,6,7.
	\item If we use transition 7 initially, nothing remains on the stack and we cannot use any transition, but the string has still not been read. Therefore we must use one of the first 6 transitions first. \\
	
	Since the first character to be read is $ a $, only the first or second transitions can be used. Since the top of the stack initially is $ \bot $, we must use transition 1. The second character to be read is $ c $. So we can only use transition 6 or 7. However we cannot use either of them as the top of the stack is A, and not $ B $ or $ \bot $ that are required to use transition 6 or 7 respectively. Since any series of transitions is unable to fully read the string and empty the stack, the machine cannot accept $ acbbbcc $.
	\end{enumerate}
\end{document}