\documentclass[11pt]{article}
\usepackage{fullpage}
\usepackage{algorithm}
\usepackage[noend]{algorithmic}
\usepackage{enumerate}
\usepackage{amsmath,amssymb,amsthm}
\usepackage{stackrel}

% Helpful Shortcuts
\newcommand{\bc}[1]{{\quad \text{(#1)}}} 	% Justification in math env
\newcommand{\st}{{\text{ such that }}} 							   % Math env
\newcommand{\abs}[1]{{ |#1 |}} 							% Absolute value / Cardinality
\newcommand{\bld}[2]{\noindent\textbf{#1:}\hspace{0.1in}#2$  $\bigskip} % Headings
\newcommand{\notimplies}{%
	\mathrel{{\ooalign{\hidewidth$\not\phantom{=}$\hidewidth\cr$\implies$}}}}
\newcommand{\linesep}{\noindent\bigskip\rule{17cm}{0.1mm}\bigskip} % Horizontal Line

% These define new environments / formats for lemmas, definitions, running time, etc.
\newtheorem{lemma}{Lemma}
\newtheorem{definition}{Definition}
\newtheorem{notation}{Notation}
\newtheorem*{claim}{Claim}
\newtheorem{observation}{Observation}
\newtheorem{conjecture}[lemma]{Conjecture}
\newtheorem{theorem}[lemma]{Theorem}
\newtheorem{corollary}[lemma]{Corollary}
\newtheorem{proposition}[lemma]{Proposition}
\newtheorem*{rt}{Running Time}

% These define nice ways to format P and OPT (use \P or \opt)
\def\P{\ensuremath{$ \mathcal{P} $}}
\def\opt{\ensuremath{\textsc{opt}}}
\renewcommand{\labelenumi}{\bf \alph{enumi}.}

\renewcommand\maketitle{
	\begin{center}
		\begin{tabular*}{6.44in}{l @{\extracolsep{\fill}}c r}
			\bfseries  &  & \bfseries CSCI 383 Spring 2019 \\
			\bfseries&  & \bfseries  Module 2 Take-home Midterm Solutions  \\
			\bfseries   &   &  \bfseries Kai Ting Keshia Yap\\ 
		\end{tabular*}
\end{center} }

\begin{document}
	\maketitle
	
	\noindent Honor Code: I affirm that I adhered to the Honor Code in this assignment. Keshia Yap\\
	
	
	\subsection*{Part 1: Grammatically Correct}
	Define $  B = \{x \mid x\in\{a,b\}^*, \#a(x) = 2\#b(x)\}. $
	\begin{enumerate}
	\item Specify a grammar $ G $ that generates $ B $.
	\item Prove that $ L(G) \subseteq  B $.
	\item Prove that $ B \subseteq L(G) $.
	\end{enumerate}
	\linesep
	\begin{enumerate}
		\item Let $ G=(N,\Sigma, P,S) $ be a CFG where 
		\begin{itemize}
			\item $ N=(S, C) $
			\item $ \Sigma=\{a,b\} $
			\item $ S\rightarrow\varepsilon \mid  SS\mid CSa\mid aSC \mid aaSb\mid bSaa $
%			\item $ S\rightarrow aSC\mid CSa\mid bSA\mid ASb\mid SS\mid \varepsilon $
			\item $ C\rightarrow ab\mid ba $
%			\item $ A\rightarrow aa $
		\end{itemize}
		Let $ \alpha $ be a sentential form derivable from $ S $ in $ G $. We define $ d(\alpha):=  \#a(\alpha)-2\#b(\alpha) $. 
		\item \begin{proof}
			We proceed by induction on the number of steps of derivation. \\
			
			\textbf{Base Case:} $ S\xrightarrow[G]{0} \alpha $. This means that $ \alpha=S $, which has 0 $ a $'s and 0 $ b $'s, so $ d(\alpha)=0 $.
			
			\textbf{Inductive Hypothesis:} Assume that for all sentential forms $ \alpha $ derivable in $ n $ steps, $ d(\alpha) = 0 $.
			
			\textbf{Inductive Case:} Assume $ S\xrightarrow[G]{n+1}\alpha $. By definition of $ \xrightarrow[G]{n+1} $, $ S\xrightarrow[G]{n}\beta S\gamma \xrightarrow[G]{1}\alpha $. \\
			
			We examine each production of $ S $ in $ G $. By the inductive hypothesis: $ d(\beta S\gamma ) = 0 $. Since $ C\rightarrow ab\mid ba $, we calculate that $ d(C)=d(b)+d(a) = -2+1=-1$.
			\begin{align*}
				 S\rightarrow \varepsilon : &&S\xrightarrow[G]{n}\beta S\gamma \xrightarrow[G]{1}\beta\gamma  : && d(\beta\gamma)=d(\beta S\gamma)=0\\
				 S\rightarrow SS : &&S\xrightarrow[G]{n}\beta S\gamma \xrightarrow[G]{1}\beta SS\gamma : && d(\beta SS\gamma)=d(\beta S\gamma) + d(S)=0+0=0\\
				 S\rightarrow aSC : &&S\xrightarrow[G]{n}\beta S\gamma \xrightarrow[G]{1}\beta aSC\gamma : && d(\beta aSC\gamma)=d(\beta S\gamma)+d(C)+d(a)=0-1+1=0\\
				 S\rightarrow CSa : &&S\xrightarrow[G]{n}\beta S\gamma \xrightarrow[G]{1}\beta CSa\gamma : && d(\beta CSa\gamma)=d(\beta S\gamma)+d(C)+d(a)=0-1+1=0\\
				 S\rightarrow aaSb : &&S\xrightarrow[G]{n}\beta S\gamma \xrightarrow[G]{1}\beta aaSb\gamma : && d(\beta CSa\gamma)=d(\beta S\gamma)+d(b)+d(aa)=0-2+2=0\\
				 S\rightarrow bSaa : &&S\xrightarrow[G]{n}\beta S\gamma \xrightarrow[G]{1}\beta bSaa\gamma : && d(\beta CSa\gamma)=d(\beta S\gamma)+d(b)+d(aa)=0-2+2=0
			\end{align*}
		
		So for all $ \alpha $ such that $ S\xrightarrow[G]{*}\alpha $ and $ \alpha\in\{a,b\}^* $, $ \alpha \in B$. So $ L(G)\subseteq B $.
		\end{proof}
		\item \begin{proof}
			Suppose $ x\in B $ and $ x\in\{a,b\}^* $. We proceed via strong induction on the length of $ x $.
			
			\textbf{Base Case:} $ \abs{x}=0 \implies x=\varepsilon$. Indeed, we have $ S\xrightarrow[G]{1}\varepsilon$.\\
			
			\textbf{Inductive Hypothesis:} Assume for all $ x\in\{a,b\}^*  $ of length $ \ell\leq n $, if $ d(x) = 0 $, then $ S\xrightarrow[G]{*} x$.\\
			
			\textbf{Inductive Case:} Assume $ \abs{x}=n+1 $, and $ x=x_1x_2\dots x_{n+1} $ where $ x_i\in\Sigma $ for all $ i\in\{1,2,\dots,n+1\} $. We can plot the graph of $ z(y(i)):=d(x_1\dots x_i) $ over $ y(i):=i $. Note: since $ x\in B $, we have that $ d(x)=0 $. Moreover, we also have $ d(\varepsilon)=0 $ trivially, so $ z(0)=0=z(n+1) $.
			\begin{itemize}
				\item \textbf{Case 1:} Suppose the graph crosses the $ y $-axis at $ 1\leq k\leq n $. Then this means there exists $ u,v\in\Sigma^* $ such that $ x=uv $ and $ d(u)=0=d(v) $, where $ \abs{u}=k $ and $ \abs{v}=n+1-k $. Since $ 1\leq k\leq n  $, $ \abs{u}, \abs{v}\leq n $ and so by the inductive hypothesis, $ S\xrightarrow[G]{*} u$ and $ S\xrightarrow[G]{*} v $. So \[S\xrightarrow[G]{1} SS \xrightarrow[G]{*} uS \xrightarrow[G]{*} uv=x.\] 
				\item \textbf{Case 2:} Suppose that the graph does not cross the $ y$-axis for all $ 1\leq i\leq n $. Then consider $ d(1) $. Since $ d(a)=1 $ and $ d(b) =-2$ and $ a,b $ are the only characters in $ \Sigma $, $ d(1)=1 $ or -2\\
				
				Suppose $ d(1)=1 $ and $ d(n)=1 $. Then $ x=aya $, and $ d(y)=-2 $, so $ y $ must contain a $ b $. Suppose $ y=\beta b\gamma $. Then by the induction hypothesis, $ S\xrightarrow[G]{*} \beta  $ and $ S\xrightarrow[G]{*} \gamma $.\\
				\[a\beta b\gamma a=x\]
				
%				then $ x=ayab, ayba, aayb, aaby, abay, abya $. So $ d(y)=d(x)-2d(a)-d(b)=0-2+2 =0$. Since $ \abs{y}=n+1-3=n-2\leq n $, we have $ S\xrightarrow[G]{*} y $. Therefore, we have 
%				\[S\xrightarrow[G]{1} aSC\xrightarrow[G]{1}aSba \xrightarrow[G]{*}ayba \quad\text{ and }\quad S\xrightarrow[G]{1} aSC\xrightarrow[G]{1} aSab\xrightarrow[G]{*}ayba.\] 
				
%				Otherwise, $ d(1)=-2 $, then $ x=byaa $ or $ baya $ or $ baay $. Similar to above, $ d(y)=d(x)-2d(a)-d(b)=0-2+2 =0$. Since $ \abs{y}=n+1-3=n-2\leq n $, we have $ S\xrightarrow[G]{*} y $. Therefore, we have 
%				\begin{itemize}
%				\item $ S\xrightarrow[G]{1} CSa \xrightarrow[G]{1}abSa \xrightarrow[G]{*}abya $ \\
%				\item $S\xrightarrow[G]{1} CSa \xrightarrow[G]{1}baSa \xrightarrow[G]{*}baya $ \\
%				\item $S\xrightarrow[G]{1} CSa \xrightarrow[G]{1}baSa \xrightarrow[G]{*}baya $
%				\end{itemize}
				
				So $ S\xrightarrow[G]{*}x $.
			\end{itemize}
		So for all $ \alpha $ such that $ \alpha\in\{a,b\}^* $ and $ \#a(\alpha)=2\#b(\alpha) $, there exists transitions such that $ S\xrightarrow[G]{*}\alpha $. So $ L(G)\supseteq B $.
		\end{proof}
	\end{enumerate}
	\newpage
	\subsection*{Part 2: Sandwiches}
	For each of the four languages below, either prove the langauge is not context-free or give an NPDA
	for the language. You do not need to prove correctness of your NPDAs, but you should include
	both an informal description of how your automaton works, as well as a formal description.
	\begin{enumerate}
	\item  Let $ b(n) $ be the number n expressed in binary. Let $ L_1 \subseteq \{0, 1, \$\}^* $ be given by
	\[ L_1 = \{b(n)\$b(n + 1)| n \geq 0\}\]
	\item  Let $ b(n) $ be the number n expressed in binary. Let $ L_2 \subseteq \{0, 1, \$\}^* $ be given by
	\[L2 = {b(n)\$b(n + 1)R | n ≥ 0},\]
	where the reverse operator applies only to $ b(n + 1) $.
	\item $ L_3 = \{wtw^R \mid w, t \in \{0, 1\}^* $ and $ \abs{w}=\abs{t}\} $
	\item $ L_4 = \{wtw^R \mid w, t \in \{0, 1\}^*\} $
	\end{enumerate}
	\linesep
	\begin{enumerate}
		\item $ L_1 $ is not context-free.
		\begin{proof} 
			We proceed by the pumping lemma. Suppose $ L_1 $ is context-free. Then there exists a CFG $ G $ in CNF with $ \ell $ non-terminals such that $ L(G)=L_1 $. Let $ k:=2^{\ell+1} $. Then the string $ 1^k0^k\$1^k0^{k-1}1\in L(G) $. Now, by the pigeon-hole principle, 
		\end{proof}
		\item $ L_2 $ is context-free. Let $ G_2=(N,\Sigma, P,S) $ be a CFG where 
			\begin{itemize}
				\item $ \Sigma=\{0,1,\$\} $
				\item $ N=\{\} $
				\item $ S\rightarrow 0\$1 \mid 1C1 \mid B1 $
				\item $ C \rightarrow 0\$1 \mid 0C0 \mid 1C1 \mid 0B1 $
				\item $ B\rightarrow 1\$0\mid 1B0  $
			\end{itemize}
		We claim that $ L_2=L(G_2) $.
		\begin{proof}
		\end{proof}
		\item $ L_3 $ is not context-free.
		\begin{proof}
			Suppose $ L_3 $ is context-free. Then there exists a CFG $ G_3 $ in CNF with $ \ell $ non-terminals such that $ L(G_3)=L_3 $. Let $ k:=2^{\ell+1} $. Then the string $ 1^{3k}0^k1^{2k} \in L_3=L(G) $, where $ w=1^{2k} $ and $ t=1^k0^k $ (clearly, $ \abs{w}=\abs{t}=2k $). Now, by the pumping lemma, for any decomposition $ x=uvzxy $ where $ \abs{vx}>0 $ and $ \abs{vzx}\leq k $, the string $ uv^izx^iy $ for any $ i\geq 0 $ is in $ L(G_3) $. So $ uv^2zx^2y\in L(G_3) $. 
			\begin{itemize}
				\item \textbf{Case 1:} 
			\end{itemize}
		\end{proof}
		\item $ L_4 $ is context-free. Let $ G_4=(N,\Sigma, P,S) $ be a CFG where 
		\begin{itemize}
			\item $ \Sigma=\{0,1\} $
			\item $ N=\{\} $
			\item $ S\rightarrow \varepsilon\mid 0S\mid 1S $
		\end{itemize}
		We claim that $ L_4=L(G_4) $.
		\begin{proof}
			It is easy to see that $ G_4 $ can produce any string of 0s and 1s of any length, so $ L(G_4)=\{0,1\}^* $. We want to show that $ \{0,1\}^* = L_4 $. Consider $ x\in L_4 $. There exists some decomposition $ x=wtw^R $. Since this string is made of 1s and 0s, it is in $ \{0,1\}^*  $. Now consider any string $ x $ in $ \{0,1\}^*  $. It is equivalent to the string $ \varepsilon x \varepsilon^R $, taking $ w=\varepsilon $, so it is in $ L_4 $.
		\end{proof}
	\end{enumerate}
	
	\newpage
	\subsection*{Part 3: False or False}
	Given a language $ A $, define $ root(A) = \{x \mid xx \in A\} $. For example, if $ A = \{aa, abc, aabb, acac, abccba\} $,
	then $ root(A) = \{a, ac\} $. Define $ perm(A) = \{x \mid x $ is a permutation of $ y \in A\} $. For example, if
	$ A = \{ab, acca\} $, then $ perm(A) = \{ab, ba, aacc, acac, acca, caac, caca, ccaa\} $. Disprove the following
	statements.
	\begin{enumerate}
	\item  If $ A $ is context free, then $ root(A) $ must be context free.
	\item If $ A $ is context free, then $ perm(A) $ must be context free.
	\end{enumerate}
	If for either part you end up using a language $ A $ which we have not proved context-free in
	class/homework/the book, give a grammar or NPDA for it, and briefly explain your construction.
	\linesep
	
	\begin{lemma}
		Context-free languages are closed under concatenation.
	\end{lemma}
	\begin{proof}
		Suppose $ L_1 $ and $ L_2 $ are languages that are context-free. Then by a theorem from class, there exists context-free grammars $ G_1=(N_1,\Sigma_1,P_1,S_1)  $ and $ G_2=(N_2,\Sigma_2,P_2,S_2) $ such that $ L(G_1)=L_1 $ and $ L(G_2)=L_2 $. Now, rename all the states of $ G_1 $ and $ G_2 $ such that they are all pairwise distinct. Create a new grammar $ G=(N,\Sigma_1\cup \Sigma_2, P, S) $ where $ P=P_1\cup P_2\cup \{S\rightarrow S_1S_2\} $. Then it is clear that if $ \alpha_1\in \Sigma_1^*$, $ \alpha_2\in \Sigma_2^*$, then \[S_1\xrightarrow[G]{*}\alpha_1\alpha_2 \text{ and }S_2\xrightarrow[G_2]{*}\alpha_2 \iff  S\xrightarrow[G]{1}S_1S_2\xrightarrow[G]{*}\alpha_1S_2\xrightarrow[G]{*}\alpha_1\alpha_2 \] 
	\end{proof}
		\begin{lemma}
		Context-free languages are closed under character replacement.
	\end{lemma}
	\begin{proof}
		Suppose the language $ A $ is context-free. Then by a theorem from class, there exists a context-free grammar $ G=(N,\Sigma,P,S) $ such that $ L(G)=A $. Given an injective function $ f: \Sigma \rightarrow (\Sigma')^*$, we can create a new grammar $ G' :=(N,\Sigma',P'S')$ where for all productions in $ P $ $ A\rightarrow \beta B \gamma $ where $ \beta,\gamma\in\Sigma $ and $ B\in N $, we add the production $ A\rightarrow f(\beta) B f(\gamma )$ to $ P' $. Then we can prove by induction on the number of steps that if $ \alpha\in(N\cup \Sigma)^* $ and $ \theta\in(N'\cup \Sigma')^* $ then 
		\[ S\xrightarrow[G]{*} \alpha \iff S'\xrightarrow[G']{*} \theta .\] This means that if $ A $ is context free, then the language $ f(A):=\{y\mid y=f(x_1)f(x_2)\dots f(x_n) \in\Sigma'^*, x=x_1x_2\dots x_n\in A, x_i\in\Sigma\forall i\in[1,n] \} $ then $f(A)= L(G') $ so $ f(A) $ is also context-free.
	\end{proof}

	\begin{enumerate}
		\item Consider the language $ L_1:=\{a^mb^na^{2n}b^\ell a^p\mid m,n,\ell,p\geq 1\} $. This is is the concatenation of languages $ L_A:=\{a^n\mid n\geq 1\} $, $ L_B:=\{b^n\mid n\geq 1\} $ and $ L:=\{b^na^{2n} \mid n\geq 1\} $ in the form $ L_1=L_ALL_AL_BL_A $. Since $ L $ is the language $ \{a^nb^{n} \mid n\geq 1\} $ under the character replacement $ a\mapsto b, b\mapsto aa $, and we showed in class that $ \{a^nb^{n} \mid n\geq 1\}  $ is context-free, $ L $ is context-free by Lemma 2. It is clear that the NPDAs $ A\rightarrow aA\mid a $ and $ B\rightarrow bB\mid b $ produce $ L_A $ and $ L_B $ respectively, so they are both context-free. By Lemma 1, $ L_1 $ is context-free. \\
		
		Alternatively, we can consider the following NPDA $ G=({S,A,B,C}, \{a,b\}, P,S) $ for $ L_1 $ where $ P $ is given by:
		\begin{itemize}
			\item $ S\rightarrow ACBA $
			\item $ A\rightarrow aA\mid a $ 
			\item $ B\rightarrow bB\mid b $
			\item $ C\rightarrow bCaa \mid baa $
		\end{itemize}
	
		So $ L_1 $ is context-free. We claim that $root(L_1)=\{a^nb^n a^n\mid n\geq 1\} $. We have shown in class that this is not context-free, so $ L_1 $ would be a counter-example to the claim \textbf{a.} above. It remains to prove that $ root(L_1)=\{a^nb^n a^n\mid n\geq 0\}  $.
		\begin{proof}
			Let $ x= a^kb^k a^k$ for some $n\geq 0$. Then $ xx=a^kb^k a^k a^kb^k a^k $ which is in $ L_1 $ where $ m=n=\ell=p=k $. So $ root(L_1)\supseteq \{a^nb^n a^n\mid n\geq 0\} $.
			
			If $ x\in root(L_1) $, then $ xx\in L_1 $. That is, there exists integers $ m,n,\ell, k\geq 1 $ such that $ xx= a^mb^na^{2n}b^\ell a^p $. 
		\end{proof}
	
		\item Consider the language $L_2:= \{(ab)^n c^n \mid n\geq 0\} $.  We see that this language is the same as the language $  \{a^n b^n \mid n\geq 0\}  $ under the character replacement $ a\mapsto ab, b\mapsto c $ Since context-free languages are closed under character replacement, $ L_2 $ is context-free. However, we claim that $ perm(L_2) $ is not context-free. 
		\begin{proof}
			Observe that if $ x\in perm(L_2) $ then $ x=(ab)^kn^kc^k $ for some integer $ k $, so $ \#a(x)= \#b(x)= \#c(x) $. Now, if $ \alpha\in\Sigma^* $ such that $k:= \#a(\alpha)= \#b(\alpha)= \#c(\alpha) $ then we can permute it to be in the form $ (ab)^kn^kc^k $, so $ \alpha\in perm(L_2) $. Therefore, \[perm(L_2) =\{\alpha\mid \#a(\alpha)= \#b(\alpha)= \#c(\alpha) \}. \] 
			
			For any $ \alpha\in\Sigma^* $, define $ d(\alpha):= \pi(\#a(\alpha)- \#b(\alpha)) - \#b(\alpha)-\#c(\alpha)$. Then
			\begin{align*}
				d(\alpha)=0 &\iff \pi(\#a(\alpha)- \#b(\alpha)) - \#b(\alpha)-\#c(\alpha) =0 &&\bc{by definition}\\
				&\iff 0=\#a(\alpha)- \#b(\alpha)=\#b(\alpha)- \#c(\alpha) &&\bc{since $ \pi $ is transcendental}\\
				&\iff  \#a(\alpha)= \#b(\alpha)= \#c(\alpha)&&\bc{by transitivity}\\
				& \iff \alpha\in perm(L_2)&&\bc{from above}
			\end{align*}
			
			Now suppose by way of contradiction that $  perm(L_2)  $ is context-free. Then there exists a CFG $ G $ in CNF with $ \ell $ non-terminals such that $ L(G)= perm(L_2) $. Let $ k:=2^{\ell+1} $. Then the string $ \alpha:=a^kb^kc^k \in  perm(L_2) =L(G) $ since it is a permutation of the string $ (ab)^kc^k $. Now, by the pumping lemma, for any decomposition $ \alpha=uvwxy $ where $ \abs{vx}>0 $ and $ \abs{vwx}\leq k $, the string $ uv^iwx^iy $ for any $ i\geq 0 $ is in $ L(G) $. \\
			
			Since $ \abs{vwx}\leq k $, the contiguous substring $ vwx $ can span at most two characters in $ \alpha $. Since $ \abs{vx}>0 $, the contiguous substring $ vwx $ cannot be empty. So the substring $ vx $ contains at least one and at most two distinct characters, so $ d(vx)\neq 0 $.\\
			
			By the pumping lemma, $ uv^2wx^2y \in L(G)$. Clearly, $ d(uv^2wx^2y) =d(uvwxy)+d(v)+d(x)=0+d(vx)\neq 0$. So $ uv^2wx^2y \not\in perm(L_2) $. Contradiction.
		\end{proof}
	\end{enumerate}
\end{document}