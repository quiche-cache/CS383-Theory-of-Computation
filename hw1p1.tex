\documentclass[11pt]{article}
\usepackage{fullpage}
\usepackage{algorithm}
\usepackage[noend]{algorithmic}
\usepackage{amsmath,amssymb,amsthm}
\usepackage{stackrel}

% These define new environments / formats for lemmas, definitions, running time, etc.
\newtheorem{lemma}{Lemma}
\newtheorem{definition}{Definition}
\newtheorem{notation}{Notation}
\newtheorem*{claim}{Claim}
\newtheorem{observation}{Observation}
\newtheorem{conjecture}[lemma]{Conjecture}
\newtheorem{theorem}[lemma]{Theorem}
\newtheorem{corollary}[lemma]{Corollary}
\newtheorem{proposition}[lemma]{Proposition}
\newtheorem*{rt}{Running Time}

% These define nice ways to format P and OPT (use \P or \opt)
\def\P{\ensuremath{$ \mathcal{P} $}}
\def\opt{\ensuremath{\textsc{opt}}}
\renewcommand{\labelenumi}{\bf \alph{enumi}.}
\newcommand{\Z}{\mathbb{Z}}
\newcommand{\linesep}{\noindent\bigskip\rule{17cm}{0.1mm}\bigskip}

\newcommand{\bc}[1]{{\quad \text{(#1)}}} 	% Justification in math env
\newcommand{\st}{{\text{ such that }}} 							   % Math env
\newcommand{\abs}[1]{{\left |#1\right |}} 							% Absolute value / Cardinality
\newcommand{\bld}[2]{\noindent\textbf{#1:}\hspace{0.1in}#2$  $\bigskip} % Headings
\newcommand{\notimplies}{%
	\mathrel{{\ooalign{\hidewidth$\not\phantom{=}$\hidewidth\cr$\implies$}}}}

\renewcommand\maketitle{
	\begin{center}
		\begin{tabular*}{6.44in}{l @{\extracolsep{\fill}}c r}
			\bfseries  &  & \bfseries CSCI 383 Spring 2019 \\
			\bfseries&  & \bfseries  Homework \#1 Solutions  \\
			\bfseries   &   &  \bfseries Kai Ting Keshia Yap\\ 
		\end{tabular*}
\end{center} }

\begin{document}
	\maketitle
	
	\noindent Honor Code: I affirm that I adhered to the Honor Code in this assignment. Keshia Yap\\
	
	\subsection*{Part  1: Equivalence Relations}
\noindent 1. Recall that an \textit{equivalence relation} is a relation, usually denoted as $ \approx $ or $ \equiv $, over a set $ \mathbb{S}, $, with the following three properties:
\begin{itemize}
	\item Reflexivity: For all $ A\in \mathbb{S}, A\approx A $.
	\item Symmetry: If $ A\approx B $ then $ B\approx A $.
	\item Transitivity: If $ A\approx B $ and $ B\approx C $ then $ A\approx C $.
\end{itemize}

In the classic road-trip game word chain, you are given an initial word $ s $ and target word $ t $,
and your goal is to build a chain of words that gets you from $ s $ to $ t $ along a series of unique
words that differ by exactly one letter. For example,  holy $ \rightarrow $ hold $ \rightarrow $ gold $ \rightarrow $ goad $ \rightarrow $
glad $ \rightarrow $ clad $ \rightarrow $ clap $ \rightarrow $ crap is a chain from holy to crap. Let $ \mathbb{W} $ be the words in the English language, and define the relation $ \approx $ on the $ \mathbb{W} $ as follows: for $ x,y\in \mathbb{W} $.

\[x\approx y \iff \text{there exists a word chain from x to y}\]

Prove that $ \approx $ is an equivalence relation on the set $ \mathbb{W} $.

\linesep

\noindent \textbf{Solution:}
\begin{itemize}
	\item Reflexivity: This is trivial as the initial word is the target word; there is a word chain of length one from the word to itself so $ x\approx x $ for all words $ x $.
	\item Symmetry: If $ x\approx y $ then there is a chain from word $ x $ to word $ y $. The chain read backwards is a chain from $ y $ to $ x $ where consecutive words differ by exactly one letter, so $ y\approx x $.
	\item Reflexivity: If $ x\approx y $ and $ y\approx z$ then there is a chain from word $ x $ to word $ y $ and a chain from word $ y $ to word $ z $. We can therefore concatenate the chain from $ x $ to $ y $ with the chain from $ y $ to $ z $ to get a chain from $ x $ to $ z $ where consecutive words differ by one letter, so $ x\approx z $.
\end{itemize}

\newpage
\noindent 2.  Recall that $ \Z $ is the set of all integers. Give an example of each of the following:
\begin{itemize}
	\item A relation on $ \Z $ that is reflexive and symmetric but not transitive.
	\item A relation on $ \Z $ that is reflexive and transitive but not symmetric.
	\item A relation on $ \Z $ that is symmetric but not reflexive.
\end{itemize}
Explain why each fails to satisfy the desired property with an example.

\linesep

\noindent \textbf{Solution:}
\begin{itemize}
%	\item Noncoprimality is a relation on $ \Z $ that is reflexive and symmetric but not transitive. \textit{Definition:} Two integers $ x,y $ are noncoprimal if they share at least one common prime divisor (there exists some prime $ p $ that divides both $ x $ and $ y $).\\

%	\item Given integers $ x,y \in \Z$, we define the relation $ x\approx y $ as ``$ x-y $ is odd''. We claim that $ \approx $ is a relation on $ \Z $ that is reflexive and symmetric but not transitive.
	\item Let $\triangle$ be a relation where $ x\triangle y $ iff $ \mid x-y\mid \leq 5 $ for all integers $ x,y $. We claim that this relation is reflexive and symmetric but not transitive.
	
	\begin{proof}
		For any integer $ x $, $ \abs{x-x}=0\leq 5 $, so the relation is reflexive. If $ x\triangle y $ then $ \abs{x-y}\leq 5 $, which implies that $ \abs{y-x}=\abs{-(x-y)}=\abs{x-y}\leq 5 $, so $ y\triangle x $. Therefore the relation is symmetric. However, $ 0\triangle 5 $ and $ 5\triangle  10$ (since $ \abs{0-5}\leq 5 $ and $ \abs{5-10}\leq 5 $), but $ \abs{0-10}\not\leq 5 $ so $ 0\not\triangle 10 $. Therefore the relation is not transitive.
	\end{proof}
	
	\item The relation $ \leq  $ is a relation on $ \Z $ that is reflexive and transitive but not symmetric. We define it as: for any two integers $ x,y $, $ x\leq y $ iff $ y-x\geq 0 $.
	\begin{proof}
		For any integer $ x $, $ x-x=0 $ so $ x\leq x $. Also, given integers $ x,y,z $ where $ x\leq y $ and $ y\leq z $, $ z-x=(z-y)+(y-x)\geq 0 $ since the non-negative integers are closed under addition. So $ \leq $ is transitive. However, $ 3\leq 5 $ but $ 5\not \leq 3 $, so $ \leq $ is not symmetric.
	\end{proof}

	\item Additive inverse is a relation on $ \Z $ that is symmetric but not reflexive.
	\begin{proof}
		We observe that for any integer $ x $, $ -x $ is the additive inverse of $ x $ and $ -(-x)=x $ is the additive inverse of $ -x $, so the relation is symmetric. However, it is not true that 5 is the additive inverse of itself, so the relation is not reflexive..
	\end{proof}
\end{itemize}


\end{document}