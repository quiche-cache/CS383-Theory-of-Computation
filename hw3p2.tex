\documentclass[11pt]{article}
\usepackage{fullpage}
\usepackage{algorithm}
\usepackage[noend]{algorithmic}
\usepackage{amsmath,amssymb,amsthm}
\usepackage{stackrel}

% These define new environments / formats for lemmas, definitions, running time, etc.
\newtheorem{lemma}{Lemma}
\newtheorem{definition}{Definition}
\newtheorem{notation}{Notation}
\newtheorem*{claim}{Claim}
\newtheorem{observation}{Observation}
\newtheorem{conjecture}[lemma]{Conjecture}
\newtheorem{theorem}[lemma]{Theorem}
\newtheorem{corollary}[lemma]{Corollary}
\newtheorem{proposition}[lemma]{Proposition}
\newtheorem*{rt}{Running Time}


\newcommand{\bc}[1]{{\quad \text{(#1)}}} 	% Justification in math env
\newcommand{\st}{{\text{ such that }}} 							   % Math env
\newcommand{\abs}[1]{{ |#1 |}} 							% Absolute value / Cardinality

% These define nice ways to format P and OPT (use \P or \opt)
\def\P{\ensuremath{$ \mathcal{P} $}}
\def\opt{\ensuremath{\textsc{opt}}}
\renewcommand{\labelenumi}{\bf \alph{enumi}.}

% Horizontal line
\newcommand{\linesep}{\noindent\bigskip\rule{17cm}{0.1mm}\bigskip}

\renewcommand\maketitle{
	\begin{center}
		\begin{tabular*}{6.44in}{l @{\extracolsep{\fill}}c r}
			\bfseries  &  & \bfseries CSCI 383 Spring 2019 \\
			\bfseries&  & \bfseries  Homework \#3 Solutions  \\
			\bfseries   &   &  \bfseries Kai Ting Keshia Yap\\ 
		\end{tabular*}
\end{center} }

\begin{document}
	\maketitle
	
	
	\noindent Honor Code: I affirm that I adhered to the Honor Code in this assignment. Keshia Yap\\
	
	\subsection*{Part  2: pU tI kcaB}
	Suppose $ A $ is languages over some alphabet $ \Sigma $. Recall that we defined a reverse operator $ R $ inductively as follows:
	\[\epsilon^R:=\epsilon, \qquad (wa^R):=a(w^R)\]
	for any $a\in \Sigma, w\in\Sigma^* $.\\
	
	We can extend this definition to languages in a natural way: let $ A^R=\{x^R\mid x\in A\} $. Using finite automata, formally prove that if $ A $ is regular then $ A^R $ is also regular. 
	
	\linesep
	
	Since $ A $ is regular, there exists a DFA $ M=\{Q,\Sigma,\delta, s, F\} $ such that $ L(M)=A $. We claim that there exists an NFA that accepts $ A^R $, which we denote $ M_R=\{Q,\Sigma,\Delta,S_R,F_R\} $. Intuitively, every string $x$ that gets accepted by $M$ has a path from the start state $s$ to some final state $f\in F$. The reverse of this path would then read in the string $x^R$ from that final state to the start state. We define:
	
	\begin{itemize}
		\item $ Q_R=Q $ 
		\item $ \Sigma $ remains the same since no letters are added/removed in the reverse of a string
		\item $ S_R = F $ 
		\item $ F_R=\{s\} $
		\item $ \Delta(q_R,a)=\{q\mid q\in Q: \delta(q,a)= q_R\} $ (i.e. reverse all arrows in $ M $)
		\item $ \hat{\Delta}(A,x)=\begin{cases}
		A & \text{if }x=\varepsilon\\
		\bigcup_{q\in\hat{\Delta}(A,w)}\Delta(q,a)&\text{if }x=wa, \text{ where }w\in \Sigma^*, a\in\Sigma
		\end{cases} $
	\end{itemize}
	
	Informally, we can think of $M_R$ as the DFA $M$ with its arrows reversed and its start states and final states swapped. We want to show that if a string $x$ is accepted by $M$, then $x^R$ would be accepted by $M_R$, i.e. $x\in L(M)\implies x^R\in L(M_R)$. This is equivalent to proving that \[\hat{\delta}(s,x) \in F \implies \hat{\Delta}(S_R,x^R)\cap F_R\neq \varnothing. \]\bigskip
	
	\newpage
	\begin{lemma}
		Let $ A\subseteq Q $. Then $ \hat{\Delta}(A,x^R)=\{q\mid q\in Q: \hat{\delta}(q,x)\in A\} $.
	\end{lemma}	
	\begin{proof}
		We proceed by structural induction. 
		
		\noindent \textbf{Base Case:} $ x=\varepsilon $
		\begin{align*}
		LHS&=\hat{\Delta}(A,\epsilon^R)= \hat{\Delta}(A,\varepsilon)&&\bc{by definition of $ R $}\\
		&=A &&\bc{by definition of $ \hat{\Delta} $}\\
		RHS&=\{q\mid q\in Q: \hat{\delta}(q,\varepsilon)\in A\}\\
		&=\{q\mid q\in Q: \hat{\delta}(q,\varepsilon)=q\in A\}&&\bc{by definition of $ \hat{\delta} $}\\
		&=\{q\mid q\in Q\cap A\}&&\bc{by set theory}\\
		&=\{q\mid q\in A\}&&\bc{since $ A\subseteq Q $}\\
		&=A
		\end{align*}
		\noindent \textbf{Inductive Hypothesis:} $ x=w\in \Sigma^*$
		\[\hat{\Delta}(A,w^R)=\{q\mid q\in Q: \hat{\delta}(q,w)\in A\}\]
		\noindent \textbf{Inductive Case:} $ x=wa$, where $ w\in \Sigma^*, a\in\Sigma $
		\begin{align*}
		\hat{\Delta}(A,(wa)^R)&=\hat{\Delta}(A,a(w^R))&&\bc{by definition of $ R $}\\
		&=\hat{\Delta}(\hat{\Delta}(A,a),w^R)&&\bc{by Part 1 of Homework 3}\\
		&=\{q\mid q\in Q: \hat{\delta}(q,w)\in \hat{\Delta}(A,a)\} &&\bc{by inductive hypothesis}\\
		&= \{q\mid q\in Q: \hat{\delta}(q,w)\in \{p\mid p\in Q: \delta(p,a)\in A\} \} &&\bc{by definition of $ \hat{\Delta }$}\\
		&= \{q\mid q\in Q: \delta(\hat{\delta}(q,w),a)\in A\}&&\bc{set theory}\\
		&=\{q\mid q\in Q: \hat{\delta}(q,wa)\in A\}&&\bc{def of $ \hat{\delta} $}\\
		\end{align*}
		\bigskip
	\end{proof}
	
	\begin{theorem}
		If $ A $ is regular, then $ A^R $ is regular.
	\end{theorem}
	\begin{proof}
		Suppose $ \hat{\delta}(s,x) \in F  $. By the above lemma, $ \hat{\Delta}(S_R,x^R)=\{q\mid q\in Q: \hat{\delta}(q,x)\in S_R=F\}\ni s $. Since $F_R=\{s\} $, the intersection of the above two sets is non-empty. Therefore,
		
		\[x\in L(M)\iff \hat{\delta}(s,x) \in F \implies \hat{\Delta}(S_R,x^R)\cap F_R\neq \varnothing\iff  x^R\in L(M_R).\]
		
		So $ L(M_R)=A^R $. We have proven in class that any NFA can be converted into a DFA. Therefore, we can convert $ M_R $ into a DFA, proving that $ A^R $ is regular.
		
		%		If a set $ \mathbb{S} $ has a non-empty intersection with $ F_R $ then the intersection must contain at least one element which is in $ F_R $.  only contains one element, $ \mathbb{S} $ must contain $ s $. 
	\end{proof}
\end{document}